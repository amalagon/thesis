\documentclass[12pt]{article}
\begin{document}
We used a cylindrical cavity cooled to $\simto 5$ K resonant at a frequency of $33.9-34.5$ GHz in the $\text{TM}_{020}$ mode. The cavity was tuned to a specific resonant frequency using a dielectric rod; we measured the resonant frequency and Q using a vector network analyzer, then recorded the amplified noise from the cavity (after being mixed down to baseband) for an hour before retuning the cavity by 3 MHz. The baseband noise when Fourier transformed is a measure of the power in each frequency interval $\delta f$ that is the sum of the power in the cavity plus additional noise from the receiver chain. 

We recorded data in twelve separate runs (Nov 19 to May 30, 2014), and measured noise for 500 settings of the cavity frequency, with an hour of integration. The actual performance of our system

\section{Procedure}

We tracked the in-phase and quadrature voltage of our baseband signal for an hour at each frequency setting. To allow for overlap between the covered frequencies, the cavity resonance was shifted by 3 MHz at a time. The dominant systematic effects were temperature variations in the cryostat. 

To reduce the effects of the the temperature shifts we discarded all data that was greater than 10 sigma away from the first minute of data. 

The data presented here were taken during 2014 Nov 19-22, Dec 05 to 07, 11 to 13, � . These runs gave a total of 500 hours of useful data. After correcting the data for the gain of the receiver and converting them to thermodynamic temperatures we compute the mean values $\Delta T_{bin}^i$ for the observations of each frequency bin of width $\delta f$ for each spectrum. The corresponding standard deviations, $\sigma_{bin}^i$ are calculated from the scatter of the individual 57 second measurements and are an estimate of the statistical uncertainty in the mean of an hour of data. ADDINBOOTSTRAP A typical value for data taken at 5 K is $\sigma_{bin189}^i = 0.25 mK$ whereas one expects $0.14$ mK from system noise alone. The excess noise at this stage is due to the $1/f$ component and in the variable temperature noise.

The final values $\bar \Delta T_{bin}^i$ for each one of the bins were obtained by weighting by $(\sigma_{bin}^i)^2$ each one of the hour observations for a given bin. The results are plotted in Figure 1 and show no statistically significant signal in any of the bins. The errors associated to each point correspond to the standard deviations estimated from the measurement statistics in the usual way. As the data will be used to decide what level of axion coupling is excluded by these measurements it is crucial to estimate the errors in Figure 1 accurately. The scatter in the values of $\Delta T_f^i$ for the different measurements could be larger than expected from the systematic effects. In principle, one could perform standard Chi-square tests for each one of the bins. Any such test would compare the scatter of the different observations of a given field with respect to the final value, with the corresponding values of $\sigma$. Let me emphasize that such a test does not depend on whether there is a true signal coming from the sky, as the scatter is computed with respect tone "local mean"; it is truly an instrumental Chi-square test. Figure 2 shows a histogram of the weighted residuals $(\Delta T - \bar \Delta T)\sigma$ for the observations. The histogram is reasonably gaussian with the a width that is consistent with what we would expect if all the scatter the values of $\Delta T$ with respect to their corresponding average value were due to fluctuations contained in $\sigma_i$. Indeed, the corresponding value of Chi-Square is of 78 with 75 degrees of freedom. Therefore, we conclude that it is legitimate to estimate the standard deviations $\bar \sigma$ in the standard way, as $\bar \sigma = [\Sigma \sigma^{-2}]^{-1/2}$. The variation of $\sigma$ frm bin to bin is mainly due to the different number of runs over which each frequency is observed. The weighted mean of the points in Figure 1 is $T_{avg} = 26 \pm 3 mk$.

It follows from visual examination of Figure 1 that none of the points differs significantly from the mean. The question is then what limits do the data place on $g_a$ from the fact that we only see the scatter shown in Figure 1. The Neyman-Pearson lemma prescribes the optimal statistical estimator for testing the hypothesis that $g_a \neq 0$. The optimal statistic turns out to be essentially a weighted Chi-square test, which can be used to find out which values of $g$ are ruled out at a certain confidence level.
\end{document}