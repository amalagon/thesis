\documentclass[11pt]{book}
\usepackage{amsmath}
\usepackage{graphicx}
\usepackage{color}
\addtolength{\oddsidemargin}{-.875in}
\addtolength{\evensidemargin}{-.875in}
\addtolength{\textwidth}{1.75in}

\addtolength{\topmargin}{-.875in}
\addtolength{\textheight}{1.75in}
\begin{document}
\chapter{Chapter 4}
\section{Detection Technique}

Despite having referred to the Primakoff effect and cavity detection of dark matter axions previously, we know set down the expressions governing the expected sensitivity and show the effective Lagrangian.

\subsection{Lagrangian}

To start from the theory, the axion appears in the Lagrangian as 
\begin{eqnarray}

[NEEDSEQN]

\end{eqnarray}

where $F_{\mu\nu}$ is the electromagnetic field tensor, and $\tilde F_{\mu\nu}$ its dual. This produces modified Maxwell's equations of the form

\begin{eqnarray}

[NEEDSEQN]

\end{eqnarray}

The effect on the microwave cavity on the signal power can be derived by decomposing the fields in the cavity as modes and plugging them back into the modified wave equation [NEEDSREF]

A physical way to see this is to rewrite the microwave cavity as an LC resonator being driven with a voltage oscillating at $m_a$ (the axion).

We assume a Maxwell-Boltzmann distribution of the axion energies, boosted by the velocity of the earth [NEEDSREF].

The form factor expresses the amount of overlap between the electric field in the cavity mode and the dark matter axion field, with the following equation:

\begin{eqnarray}

[NEEDSEQN]

\end{eqnarray}

The form factor is zero except for modes of the form $TM_{0n0}$, and goes as 1/$n^2$, so sticking to modes with low index is favorable.

The two approaches yield the same result: the energy in the cavity on resonance as a result of the axion photon conversion is:

\begin{eqnarray}

[NEEDSEQN]

\end{eqnarray}

The power is given by $P = U\omega/Q$ and so the power expected is

\begin{eqnarray}

[NEEDSEQN]

\end{eqnarray}

Re-written in terms of the laboratory parameters, this gives us an expected signal power of $10^{-17}$ W coming out of the cavity for the following experimental values:

\begin{eqnarray}

[NEEDSEQN]

\end{eqnarray}

From the sensitivity expression one can see that maximizing the magnetic field will yield the greatest improvement. The value of the quality factor can be analytically calculated; it is a function of the skin depth and frequency, and goes as:

\begin{eqnarray}

[NEEDSEQN]

\end{eqnarray}

In our experiment we operate at frequencies of 34 GHz. Our empirical loaded quality factors are close to the analytical prediction.

\end{document}