\documentclass[11pt]{book}
\usepackage{amsmath}
\usepackage{graphicx}
\usepackage{color}
\begin{document}

\chapter{Chapter 2}
\section{Dark Matter ALPs}

There is strong astrophysical evidence that a large portion of the matter in our galaxy is non-luminous. Measurements of temperature fluctuations in the Cosmic Microwave Background suggest that this dark matter must be non-relativistic today [NEEDSREF]. The argument for non-relativistic matter today usually relies on thermalization arguments; which imply that the heavier the matter, the cooler, or more non-relativistic it is today. However, there are other mechanisms by which very light particles could be produced in the early universe such that they are non-relativistic today.

\subsection{$U(1)_{PQ}$ symmetry}

To describe this mechanism we first have to describe the mechanism of symmetry breaking which gives rise to light particles. To do so, we will go through the process which gives rise to the canonical axion. This process was postulated by Peccei and Quinn in 1977 [NEEDSREF] as a natural way to produce CP conservation the theory of strong interactions, Quantum Chromodynamics. 

The Peccei-Quinn solution to this strong CP problem, as it is known, was to introduce a global chiral U(1) symmetry, denoted $U(1)_{PQ}$. This symmetry can be spontaneously broken at some energy scale $f_a$, producing massless Goldstone bosons from the azimuthal degree of freedom (this is the axion), and some high energy mode from the radial degree of freedom; this was deemed to be the two Higgs doublets in the original theory, which sets the energy scale, but there are many models that construct other scalars as the radial mode, allowing the energy scale to decouple from the electroweak scale [NEEDSREF].

The axion acquires mass through explicit symmetry breaking of the$U(1)_{PQ}$ symmetry with QCD. This occurs because of a chiral anomaly [NEEDSREF]. In all respects this is analogous to the process by which the pion, the Goldstone boson of the axial $U(1)_A$, acquires its mass through a chiral anomaly. The mass that the axion acuqires is proportional to $m_a \sim \Lambda_{QCD}^2/f_a$. The axion coupling to matter is inversely proportional to the energy scale $f_a$ and thus proportional to the mass.

Using Higgs doublets would give one a rough prediction for the axion mass of 150 keV; this axion would have a lifetime of 1 microsecond and would have been detected in collider experiments through quarkonium decays and beam dump experiments [NEEDSREF]. The branching ratios observed were significantly less than predicted; thus this axion was ruled out.

However, as stated above, the symmetry can come from other scalars; which would separate $f_a$ from the electroweak scale and let it be much higher, and so the axion coupling and mass would be much weaker. In 1983 Pierre Sikivie proposed using the axion to two photon coupling [NEEDSREF] to search for axions, and proposed that axions would make good dark matter candidates. This interaction is the Primakoff effect, which can be written effectively as $E \dot B$. With strong magnetic fields, the axion to two photon transition is made much stronger.

Note, I say axion here, but these searches apply for axion-like particles in general. The only difference is that axion-like particles do not have the relationship between coupling and mass, so there are two free parameters in the theory.

In order for ALPs to be good dark matter candidates, they must have the correct abundance. We now go over how ALPs could be produced in the early universe such as to have the correct density today.

\subsection{Misalignment Mechanism}

We have covered how, by the addition of a global symmetry that has explicit symmetry breaking at some high energy scale, a low mass boson can arise. The mass of this particle will "turn on" when the energy scale at which the symmetry breaking occurs is matched in the energy scale of the universe (or more commonly described, the temperature). If we think of the initial 2$\pi$ degrees of freedom of the azimuthal mode, a simple picture would be of the tipping of the $\phi^4$ potential, or the wine bottle potential, leading to the angle to settle at one particular value. As the angle $\theta$ must approach its minimum, there will be oscillations around that point. These oscillations will be affected by the expansion of the universe; one can write a damped harmonic oscilllator equation of motion [NEEDSREF] where H is the hubble parameter.

As long as $\theta$ is not too small, these damped oscillations will result in a present day energy density roughly equal to the dark matter density we observe [NEEDSREF].

The cosmological history is also affected by whether the mass is turned on before or after inflation. With the 2014 BICEP result [NEEDSREF], we can set the scale of inflation at $10^{14} GeV$. If the axion acquires mass before inflation, then there is a uniform axion mass in our horizon; otherwise, there might be several causal patches with axion masses at different values, and we can only average over them to get a predicted value [NEEDSREF].

\end{document}