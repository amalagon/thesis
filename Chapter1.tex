\documentclass[12pt,twosides]{book}
\usepackage{amsmath}
\usepackage{graphicx}
\usepackage{color}
\begin{document}

\section{abstract}

One of the most basic open questions in physics today concerns the nature of dark matter. There has been strong observational evidence for the existence of dark matter for nearly eighty years, yet we still know little about dark matter, despite the fact that it composes $23\%$ of our universe. One leading candidate for dark matter is the axion, a light pseudoscalar particle. Searches for the axion have been experimentally challenging, as the axion's coupling to ordinary matter is very weak. One technique that is extremely sensitive to dark matter axions is the method of resonant detection using microwave cavities and the Primakoff effect to convert axions to photons in a strong magnetic field. This method has been applied successfully to rule out models of axions from 1.9 to 3.5 $\mu$eV. However, prior to this work , the technique had not been applied to look for axions of higher mass in the 0.1 to 1 $meV$ range. It is interesting to search for axions in this mass range because it is complementary to current axion microwave cavity searches, because there are several hints that the favored axion mass is in this range, and because the recent BICEP results constrain the axion mass in a simple cosmological history to be within that region. We present here the first microwave cavity search for dark matter axion-like particles in this region. By tuning a cryogenically cooled cavity and measuring the $TM_{020}$ mode in a 7 Tesla background magnetic field, we look axion-like particles (if they form part of the dark matter)  tranforming to photons via the Primakoff effect. When the photon is on resonance with the microwave cavity, enhancement of the signal power results. Cryogenic amplifiers and a low noise receiver chain allow us to reduce our background from thermal noise. In our experiment, we ran for four months and swept the microwave cavity resonant frequency from 33.9 to 34.5 GHz, corresponding to an axion mass of 140.2  to 142.7 $\mu$eV. We are also able to set limits on hidden photon coupling to matter if these vector bosons are responsible for dark matter. As with first runs of microwave cavity detectors in the lower mass region, there are several technical challenges that must be solved in order to reach the sensitivity to observe or exclude canonical axion models. However, as we do not know where the axion mass is it is valuable to develop experiments and techniques to search for them in their entire possible mass range. If observed, axion (and ALP) dark matter would not only be an important advance in our knowledge of dark matter but also give us clues about processes at high energy scales inaccessible by other methods.

\tableofcontents
\section{Acknowledgments}

\chapter{Background and Motivation}

%\subsection{The Nature of Dark Matter, Physics at High Energy Scales}

One of the most basic open questions in physics today concerns the nature of dark matter. Nearly eighty years ago, strong observational evidence was presented that non-luminous, gravitationally interacting matter exists \cite{zwicky37}; since then the existence of matter that does not interact electromagnetically has been shown, with strong evidence in the rotation curves of spiral galaxies \cite{rubin80}, and precision measurements of the Cosmic Microwave Background can measure the dark matter density in our universe to be 23$\%$ \cite{planck14}. However, to date we know little more than the fact that this dark matter is most likely a stable, neutral particle that is non-relativistic and non-baryonic [NEEDSREF]\footnote{although dark matter could be caused by modifications to gravity}. There are several strongly motivated candidates for dark matter; the three most promising today are sterile neutrinos \cite{kusenko09}, Weakly Interacting Massive Particles (WIMPs) [NEEDSREF], and axions. 

WIMPs have historically been the most popular dark matter candidate; they are particles that have weak scale cross-sections with ordinary matter. The theoretical support for WIMPs comes from Supersymmetric theories (SUSY), which solves what is known as the gauge hierarchy problem, or the fact that the Higgs mass is quadratically divergent. By introducing partners for all the known partners with opposite spin-statistics, SUSY solves this divergence problem. The lightest SUSY particle, the neutralino, would be a WIMP. Various experiments have searched for WIMPs, either directly through the recoil of nuclei when WIMPs strike them \cite{lux14}, indirectly through their annihilation with each other in the galaxy \cite{slatyer09}, as well as searches in colliders \cite{rajaraman11}. The null results from these experiments suggest that it is worthwhile investigating all dark matter candidates. Moreover, dark matter could be formed from any or all of these particles. This work focuses on a dark matter axion-like particle search. Axion-like particles (ALPs) are light bosons that have similar interactions and properties as the axion, but do not arise from the same theory.

The work done as part of the Yale Microwave Cavity Experiment (YMCE) was a search for dark matter pseudoscalars that couple to two photons, also known axion-like particles (ALPs). This project is part of YMCE's work in constraining exotic particles in the 140 $\mu$eV mass region using cryogenically cooled microwave cavities, with previous projects being a search for dark matter scalars, and one looking for photon regeneration due to hidden photons. This mass region is so far unexplored by other microwave cavity experiments, as it is more challenging to reach the sensitivities needed to detect the theorized axion.

This work describes the dark matter search; the design of the cavity, operation of the experiment, and measurements taken. The remainder of this chapter discusses the motivation for axion (and ALP) dark matter in more detail. Chapter 2 describes the technique of using microwave cavities, while Chapter 3 provides some background on astrophysics and cosmology necessary for understanding the current parameter space. Chapter 4 describes the experiment, while Chapters 5, 6, and 7 focus on the microwave cavity design and data analysis, which I worked on. This includes simulations of the cavity coupling and itsform factor, tuning, and Q response at cryogenic temperatures (Chapter 5), the construction of a data analysis pipeline and determining an exclusion limit (Chapter 6), and work on the hardware automation for the experiment (Chapter 7).

\section{Why axions?}

We briefly review motivations for the axion and axion-like particles (ALPs) as a dark matter candidate. For a more comprehensive discussion see \cite{hewett12}, \cite{arias12}, and \cite{kim87}. 


 \textbf{strong CP problem} \hfill \\

The motivation for axions, or QCD axions, is very strong. The axion arises from a mechanism introduced to explain the puzzling fact that parity and time violation are not observed in the theory of strong interactions, Quantum Chromodynamics (QCD). This is odd, as there is a term in the QCD Lagrangian that is explicitly P- and T- violating:
\begin{align*}
L = \bar \theta g^2 G \tilde G
\end{align*}

Although there is a free parameter, $\bar \theta$ in this P, T (or CP- if you believe in the CPT Theorem) violating term that can be set to zero, this constitutes a fine tuning problem as the observable $\bar \theta$ is the sum of two independent terms from two different sectors, which should not a priori cancel each other. This "strong CP problem" can be solved by introducing a new global, chiral symmetry, as proposed by Peccei and Quinn in 1977 \cite{peccei77}. The spontaneous breaking of this symmetry at some energy scale $f_a$ (that the symmetry must be broken is due to the non-vanishing quark masses) introduces the axion as the resulting massless Goldstone boson. Due to a chiral anomaly with QCD, or to rephrase due to instanton effects, the axion experiences a potential, which causes it to acquire a mass proportional to the energy scale of QCD, $\Lambda_{QCD}$ and inversely proportional to the energy scale of the symmetry breaking $f_a$ \cite{weinberg78}, \cite{wilczek78}. The role of anomalies is widespread in physics; a chiral anomaly with electromagnetic fields solves the U(1) problem, which is the question of why the $\eta\'$ is not degenerate with the pion mass. It is due to anomalies that this occurs. For more comprehensive treatment of anomalies, see \cite{bardeen07}. The axion coupling constant to matter ends up also being inversely proportional to $f_a$, so the mass and coupling constant of the axion are directly proportional. The axion coupling to gluons implies a generic coupling to photons, through mixing with the pion; although there is some model dependence for the coupling and it can be set to zero, this represents another fine tuning problem.

The term axion-like particles (ALPs) describes bosons that, like the axion, acquire a mass through some explicit symmetry breaking of a new symmetry, although not necessarily due to an interaction with QCD. Familons and majorons are two such examples of ALPs \cite{kim87}, but they arise generically in string theories and extensions to the standard model, as the mechanism of an additional U(1) symmetry is a minimal extension, and anomalies must often take place to cancel divergences. For ALPs, the coupling and mass are no longer necessarily connected, so there are two free parameters to their theory. However, the idea of a light, spinless, neutral, chargeless boson is common for both axions and ALPs.

 \textbf{Models of the axion} \hfill \\

The first model of the axion came from introducing the new symmetry through two Higgs doublets, as proposed by Peccei and Quinn. The two doublets, $\lambda_1$ and $\lambda_2$ combine to have a non-zero vacuum expectation value. The phase of the vacuum expectation value is the axion; by current algebra techniques that mass of the axion can be computed to be 

\begin{align*}
m_a = \frac{Nm_\pi f_\pi}{v_F}\frac{\sqrt{z}}{1+z}(\frac{1}{x}+x) = (23 \text{keV})(\frac{1}{x}+x)
\end{align*}

for $ x = <\lambda_1>/<\lambda_2>$ is the ratio of the vacuum expectation values of the doublets and $z = m_u/m_d$ is the ratio of the up and down quark masses.

This Peccei-Quinn-Weinberg-Wilczek axion was predicted to have a mass of some keV, but was ruled out in reactor and accelerator experiments \cite{crystalball90} [NEEDSREF].

Simple extensions to the Peccei Quinn mechanism modified the energy scale of the symmetry breaking, decoupling it from the electroweweak scale, and allowing the axion mass to be much lighter. The Dine, Fischler, Srednicki, Zhitniskii model (DFSZ) \cite{dine81},\cite{zhitniskii81}, adds a single heavy scalar in addition to the Higgs doublets introduced before. 

The mass of the axion is then calculated to be:

\begin{align*}
m_a = \frac{f_\pi}{f_a} m_\pi N \frac{\sqrt{z}}{1+z}
\end{align*}

For $z= 0.56$, $f_\pi = 93 \text{ MeV}$, and $m_\pi = 135\text{ MeV}$, $m_a \approx \frac{10^7\text{GeV}}{f_a}eV$.

Another model, the Kim, Vainshtein, Shiman, Zakharov model \cite{kim79},\cite{shifman80}, extends the Peccei Quinn mechanism by introducing a new heavy quark and complex scalar field, as well as a discrete symmetry. The axion then couples directly to this heavy quark and through the quark, to ordinary matter. The axion mass is then:

\begin{align*}
m_a = \frac{\sqrt{z}}{1+z}m_\pi\frac{f_\pi}{f_a}
\end{align*}

and the coupling to two photons is:

\begin{align*}
g_{a\gamma\gamma}^{KSVZ} &= \frac{\alpha}{2\pi f_a}(\frac{E}{N}-\frac{2}{3}\frac{4+z}{1+z}) 
\\ &= 1.93\times10^{-15}\text{GeV}^{-1}(\frac{E}{N}-\frac{2}{3}\frac{4+z}{1+z})(\frac{m_a}{10^{-5}\text{eV}})
\end{align*}

In conclusion, for the two photon interaction for which $\mathscript{L} = g_{a\gamma\gamma}E\dotB a$, the coupling is

\begin{align*}
g_{a\gamma\gamma} &= -\frac{3\alpha}{8\pi f_a}\xsi = - \frac{m_a/\text{eV}}{0.69\times10^{10}\text{ GeV}}\xsi 
\\ \xsi &= \frac{4}{3}(\frac{E}{N} - 1.92 \pm 0.08)
\end{align*}

$E/N$ is the ratio of the color to electromagnetic anomalies. For the DFSZ mode it is 8/3, for the KVSZ model it is 0. It cannot be guaranteed that $E/N$ does not equal 2, however, in which case the coupling to photons would be severely suppressed.


\section{Detecting axions}

Astrophysical arguments can constrain the axion mass severely. The requirement is that axions, which can produced in the cores of stars and act as an additional energy loss channel, have couplings lower than those which would produce a conflict with observation. The limits on the coupling strength to photons for globular clusters and the limit on the coupling strength to nucleons in the case of supernova SN1987A can be translated into limits on the axion mass of $m_a \leq 10^{-2} \text{eV}$ or $f_a \geq 10^9\text{GeV}$ \cite{raffelt08}.

I briefly review the techniques used to search for axions:

\begin{description}

\item \textbf{Solar axions} \hfill \\

Here one uses the Primakoff effect to search for axions produced in the sun due to the strong electric and magnetic fields in the plasma. By aiming a dipole magnet at the sun, the axions streaming from the sun can reconvert into photons (x-rays) and then be detected. The main experiment here is CAST \cite{cast11}, which has begun to edge out the globular cluster limit although not completely.  As \cite{raffelt08} notes, because the massive axions are converting to massless photons, there is a large momentum mismatch. The massive axion as momentum $k_a = (\omega^2 - m_a^2)^{1/2}$ while the photon has momentum $k_\gamma = \omega$. The difference is $q = k_\gamma - k_a \approx m_a^2/2\omega$ for $m_a \ll \omega$, so the length over which one can coherently detect a signal is limited, so conversion is a function of length.

\item{Bragg diffraction}
Another experiment used the electric fields of crystals to convert 

\item{Vacuum Birefringence}

\item{Rydberg Atoms}

\item{Photon Regeneration}


\item{Mossbauer Absorption}

\item{Oscillating EDMs, Dish Antennas}

\item{RF cavity}

While in the helioscopes the conversion was proportional to the length of the detector, for axions traveling along the magnetic field, for axion to photon transitions in the presence of a background magnetic field with high-Q cavities, the conversion probability depends on the overalp of the magnetic field and the electric field of the cavity mode, so thus the volume.

\subsection{Why microwave cavities?}
The work in this thesis describes a search for dark matter axion-like particles, using a cryogenically cooled microwave cavity in a strong magnetic field. Methods to search for these dark matter particles have been ongoing for the last four decades; however, since dark matter candidates by definition interact very feebly with ordinary matter, searching for any of these particles is experimentally challenging.  As this thesis is concerned with a dark matter axion search, I will briefly review the techniques



While I will go into more detail about each of the candidates later, axions in particular also solve a problem in particle physics related to symmetry violations. 




At this point, it is firmly established that a large portion of the matter in the universe is non-luminous. Through astrophysical observations, we can estimate that the mass to light ratio in spiral galaxies is much higher than we would expect from the amount of luminous matter alone [NEEDSREF]. The existence of this gravitationally interacting matter influences the rate of structure formation in the early universe, and measurements of the cosmic microwave background can state that the observations are consistent with a $\Lambda$CDM model; that is, a model of the universe as composed of non-baryonic, non-relativistic matter [NEEDSREF]. We know very little more about this dark matter other than the conclusions stated above.

At the same time, the nature of physics at high energy (> TeV) scales is unknown. The equations that make up the Standard Model do not extend into this energy range, and current experiments that can produce physics at high energies (the LHC, other colliders) will be limited by the size of the experiments to explore energies of at most 10 TeV. Ther are many theories of physics at high energy scales; all of these predict new particles at higher energies, or additional symmetries. The high-energy particles or new effects at high energies can enter as small corrections in loop diagrams, leading to small corrections in low-energy observables. The electric dipole moment is one such observable that can be measured in laboratories but whose value is sensitive to high energy physics. The most recent measurement of the electric dipole moment of the electron \cite{acme14} done by the ACME experiment has seen a result consistent with zero, ruling out parameters of 100 TeV theories such as Supersymmetric models. These Supersymmetric (SUSY) theories are high energy theories that solve problems in high energy particle physics, and also predict a stable particle that is a very good candidate for the particle of cold dark matter.

Another probe of this high energy physics is to look for extremely light, feebly interacting particles. These particles arise as the low-mass boson accompanying the breaking of new symmetries at high energies (usually >$10^{14}$ eV). These particles, known generally as Weakly Interacting Sub-eV Particles (WISPs), are good cold dark matter candidates as well, granted that their abundance can be accounted to match the present abundance of dark matter. A particular particle belonging to the WISP category, are axion-like particles (ALPs). This denotes pseudoscalar spin 0 bosons with a non-zero coupling to two photons. ALPs take their name from the axion, a particle with the properties described above which arises from a new symmetry which was postulated by Peccei and Quinn \cite{peccei77} to solve the problem of charge and parity conservation in the theory of strong interactions when a priori there is no physical reason this CP conservation should hold.

As direct detection searches for  the SUSY dark matter candidate with no results, there is an intrinsic neutrino background that will soon limit the experiments' sensitivity within another few orders of magnitude from where they can presently search. In addition, no particles have been seen at the LHC, which expected to find new particles that would be SUSY particles. The EDM experiments have constrained different SUSY models with no observation of the electric dipole moment. At this time, it seems necessary to build small tabletop experiments that can look for other dark matter candidates, such as axion-like particles. There is a wide space in which to search for these WISPs, and having measurements allows us to constrain models. A discovery would give us clues as to the nature of physics at high energy scales, and tell us about the composition of dark matter.

\subsection{The State of Dark Matter Searches}

Starting from the assumption that dark matter is a particle, people look for heavy, stable, neutral particles, known by the acronym Weakly Interacting Massive Particles (WIMPs), under which the SUSY candidate falls. People also look for the WISPs. There are direct and indirect detection searches going on for both candidates. For WIMPS, direct searches look for WIMP to strike a nucleus, causing it to recoil. The indirect searches assume that WIMPs annihilate with each other, being their own antiparticle, and look for signatures of these events in the galaxy.

For WISPs, direct searches primarily use the ALP to two photon vertex, as it is easy to produce large electromagnetic fields in the laboratory and easy to detect photons. Searches assuming that the ALP is the dark matter are only using microwave cavities as resonant detectors \cite{admx10} although a Rydberg atom detection experiment was performed \cite{yamamoto00}. These experiments depend on the dark matter density at Earth. There are also new ideas involving the axion as inducing an oscillating electron dipole moment in nucleons \cite{budker13}, which will be sensitive to  ALPs of a different mass than those in the microwave cavity experiments. There are other experiments looking for axion-like particles, although they do not use dark matter as the source of the WISPs. These involve looking for axions produced in the sun (CAST) \cite{cast11} or directly producing axions from the two photon vertex using a laser and strong magnetic field (ALPS-I, ALPS-II) \cite{ehret10}. The direct production or light-shining-through-wall experiments has the least number of assumptions, but these experiments (solar, microwave cavity, lsw) together form a complementary way of searching for ALPs with the two-photon vertex.

There are also astrophysical arguments that can limit the coupling of the axion to the two photon vertex. The ALP, produced by various processes in red giants, neutron stars, and white dwarfs, would radiate energy from the star, altering the stellar evolution and thus stellar lifetime \cite{turner89}. By observing stellar lifetime consistent with 10$\%$, constraints on the axion coupling can be derived\cite{raffelt95}. A strong limit comes from the observation of neutrinos from the supernova explosion of SN1987A. The observation of neutrinos number and duration consistent with expectations limits the role of axions in acting as a energy loss channel. The limit is on the axion to nucelon coupling, but can be related to the axion to two photon coupling.

This dissertation will focus on a direct search for dark matter axions using microwave cavities as resonant detectors over the frequency range 33.9 to 34.5 GHz.

\subsection{Outline}

This dissertation will describe the pilot run of the Yale Microwave Cavity Experiment (YMCE) to look for dark matter ALPs in the mass range 140.2-142.7 $\mu$eV. The run set limits on the ALP-two photon coupling with sensitivity slightly better than the best previous limit set by CAST. The experiment used a microwave cavity that was tuned, immersed in a strong magnetic field - over the cavity bandwidth we would be sensitive to photons whose energy is equal to the incoming dark matter axion energy.  From the data taken, the analysis excludes axion-like particles with two-photon coupling $g_{a\gamma\gamma} < 8.6$ $1/GeV$. We end by suggesting future directions for the experiment.
The outline of the dissertation will be as follows:

\textbf{Chapter 2: Dark Matter ALPs} describes the mechanism by which these light ALPs can have the relic abundance today to match dark matter abundances observed, as well as briefly describing the specific symmetry breaking mechanism by which the axion comes about to solve the strong CP problem.

\textbf{Chapter 3: Parameter Space} goes over the current bounds on axion coupling and mass, and where microwave cavity searches fit into this field.

\textbf{Chapter 4: Detection Technique} Explains the sensitivity we can achieve using microwave cavities in strong magnetic fields, and how the expression for signal power determines what we optimize in the experiment

\textbf{Chapter 5: YMCE Experiment} goes through the components of the experiment, design of the microwave cavity, and construction of the receiver. We also go through the data taking process and summary of data taken.

\textbf{Chapter 6: Thermal Noise} The background for the experiment comes from thermal noise; we describe the expected background.

\textbf{Chapter 7: Data Analysis} puts down the analysis chain that takes the raw time domain voltage measurements and turns them into average power spectra. 

\textbf{Chapter 8: Axion Signals} We return to the expected axion power expression and walk through how that is modified when the axion energy is off resonance from the cavity resonance, but still within the cavity bandpass.

\textbf{Chapter 9:  Limits} We describe the fits and cuts applied to look for excesses in single bins that would be a hint of axion photon conversion.

\textbf{Chapter 10: Results} We present the upper bound on the ALP to two photon coupling for the mass range investigated.

\textbf{Chapter 11: Future Work} we conclude by an outlook for future work on YMCE.

\bibliographystyle{plain}
\bibliography{thesisbib}
\end{document}