\documentclass[11pt]{book}
\usepackage{amsmath}
\usepackage{graphicx}
\usepackage{color}
\begin{document}
\tableofcontents

\chapter{Theory}
\section{Introduction}
\subsection{The Nature of Dark Matter, Physics at High Energy Scales}

At this point, it is firmly established that a large portion of the matter in the universe is non-luminous. Through astrophysical observations, we can estimate that the mass to light ratio in spiral galaxies is much higher than we would expect from the amount of luminous matter alone [NEEDSREF]. The existence of this gravitationally interacting matter influences the rate of structure formation in the early universe, and measurements of the cosmic microwave background can state that the observations are consistent with a $\Lambda$CDM model; that is, a model of the universe as composed of non-baryonic, non-relativistic matter [NEEDSREF]. We know very little more about this dark matter other than the conclusions stated above.

At the same time, the nature of physics at high energy (> TeV) scales is unknown. The equations that make up the Standard Model do not extend into this energy range, and current experiments that can produce physics at high energies (the LHC, other colliders) will be limited by the size of the experiments to explore energies of at most 10 TeV. Ther are many theories of physics at high energy scales; all of these predict new particles at higher energies, or additional symmetries. The high-energy particles or new effects at high energies can enter as small corrections in loop diagrams, leading to small corrections in low-energy observables. The electric dipole moment is one such observable that can be measured in laboratories but whose value is sensitive to high energy physics. The most recent measurement of the electric dipole moment of the electron [NEEDSREF] done by the ACME experiment has seen a result consistent with zero, ruling out parameters of 100 TeV theories such as Supersymmetric models. These Supersymmetric (SUSY) theories are high energy theories that solve problems in high energy particle physics, and also predict a stable particle that is a very good candidate for the particle of cold dark matter.

Another probe of this high energy physics is to look for extremely light, feebly interacting particles. These particles arise as the low-mass boson accompanying the breaking of new symmetries at high energies (usually >$10^{14}$ eV). These particles, known generally as Weakly Interacting Sub-eV Particles (WISPs), are good cold dark matter candidates as well, granted that their abundance can be accounted to match the present abundance of dark matter. A particular particle belonging to the WISP category, are axion-like particles (ALPs). This denotes pseudoscalar spin 0 bosons with a non-zero coupling to two photons. ALPs take their name from the axion, a particle with the properties described above which arises from a new symmetry which was postulated by Peccei and Quinn [NEEDSREF] to solve the problem of charge and parity conservation in the theory of strong interactions when a priori there is no physical reason this CP conservation should hold.

As direct detection searches for  the SUSY dark matter candidate with no results, there is an intrinsic neutrino background that will soon limit the experiments' sensitivity within another few orders of magnitude from where they can presently search. In addition, no particles have been seen at the LHC, which expected to find new particles that would be SUSY particles. The EDM experiments have constrained different SUSY models with no observation of the electric dipole moment. At this time, it seems necessary to build small tabletop experiments that can look for other dark matter candidates, such as axion-like particles. There is a wide space in which to search for these WISPs, and having measurements allows us to constrain models. A discovery would give us clues as to the nature of physics at high energy scales, and tell us about the composition of dark matter.

\subsection{The State of Dark Matter Searches}

Starting from the assumption that dark matter is a particle, people look for heavy, stable, neutral particles, known by the acronym Weakly Interacting Massive Particles (WIMPs), under which the SUSY candidate falls. People also look for the WISPs. There are direct and indirect detection searches going on for both candidates. For WIMPS, direct searches look for WIMP to strike a nucleus, causing it to recoil. The indirect searches assume that WIMPs annihilate with each other, being their own antiparticle, and look for signatures of these events in the galaxy.

For WISPs, direct searches primarily use the ALP to two photon vertex, as it is easy to produce large electromagnetic fields in the laboratory and easy to detect photons. Searches assuming that the ALP is the dark matter are only using microwave cavities as resonant detectors [NEEDSREF], although a Rydberg atom detection experiment was performed [NEEDSREF]. These experiments depend on the dark matter density at Earth. There are also new ideas involving the axion as inducing an oscillating electron dipole moment in nucleons [NEEDSREF], which will be sensitive to  ALPs of a different mass than those in the microwave cavity experiments. There are other experiments looking for axion-like particles, although they do not use dark matter as the source of the WISPs. These involve looking for axions produced in the sun (CAST) [NEEDSREF] or directly producing axions from the two photon vertex using a laser and strong magnetic field (ALPS-I, ALPS-II) [NEEDSREF]. The direct production or light-shining-through-wall experiments has the least number of assumptions, but these experiments (solar, microwave cavity, lsw) together form a complementary way of searching for ALPs with the two-photon vertex.

There are also astrophysical arguments that can limit the coupling of the axion to the two photon vertex. The ALP, produced by various processes in red giants, neutron stars, and white dwarfs, would radiate energy from the star, altering the stellar evolution and thus stellar lifetime [NEEDSREF]. By observing stellar lifetime consistent with 10$\%$, constraints on the axion coupling can be derived [NEEDSREF]. A strong limit comes from the observation of neutrinos from the supernova explosion of SN1987A. The observation of neutrinos number and duration consistent with expectations limits the role of axions in acting as a energy loss channel. The limit is on the axion to nucelon coupling, but can be related to the axion to two photon coupling.

This dissertation will focus on a direct search for dark matter axions using microwave cavities as resonant detectors over the frequency range 33.9 to 34.5 GHz.

\subsection{Outline}

This dissertation will describe the pilot run of the Yale Microwave Cavity Experiment (YMCE) to look for dark matter ALPs in the mass range 140.2-142.7 $\mu$eV. The run set limits on the ALP-two photon coupling with sensitivity slightly better than the best previous limit set by CAST. The experiment used a microwave cavity that was tuned, immersed in a strong magnetic field - over the cavity bandwidth we would be sensitive to photons whose energy is equal to the incoming dark matter axion energy.  From the data taken, the analysis excludes axion-like particles with two-photon coupling $g_{a\gamma\gamma} < 8.6$ $1/GeV$. We end by suggesting future directions for the experiment.
The outline of the dissertation will be as follows:

\textbf{Chapter 2: Dark Matter ALPs} describes the mechanism by which these light ALPs can have the relic abundance today to match dark matter abundances observed, as well as briefly describing the specific symmetry breaking mechanism by which the axion comes about to solve the strong CP problem.

\textbf{Chapter 3: Parameter Space} goes over the current bounds on axion coupling and mass, and where microwave cavity searches fit into this field.

\textbf{Chapter 4: Detection Technique} Explains the sensitivity we can achieve using microwave cavities in strong magnetic fields, and how the expression for signal power determines what we optimize in the experiment

\textbf{Chapter 5: YMCE Experiment} goes through the components of the experiment, design of the microwave cavity, and construction of the receiver. We also go through the data taking process and summary of data taken.

\textbf{Chapter 6: Thermal Noise} The background for the experiment comes from thermal noise; we describe the expected background.

\textbf{Chapter 7: Data Analysis} puts down the analysis chain that takes the raw time domain voltage measurements and turns them into average power spectra. 

\textbf{Chapter 8: Axion Signals} We return to the expected axion power expression and walk through how that is modified when the axion energy is off resonance from the cavity resonance, but still within the cavity bandpass.

\textbf{Chapter 9:  Limits} We describe the fits and cuts applied to look for excesses in single bins that would be a hint of axion photon conversion.

\textbf{Chapter 10: Results} We present the upper bound on the ALP to two photon coupling for the mass range investigated.

\textbf{Chapter 11: Future Work} we conclude by an outlook for future work on YMCE.

\end{document}