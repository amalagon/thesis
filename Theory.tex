\documentclass[11pt]{book}
\usepackage{amsmath}
\usepackage{graphicx}
\usepackage{color}
\begin{document}
\tableofcontents

\chapter{Theory}
\section{Introduction}
\subsection{The Nature of Dark Matter, Physics at High Energy Scales}

At this point, it is firmly established that a large portion of the matter in the universe is non-luminous. Through astrophysical observations, we can estimate that the mass to light ratio in spiral galaxies is much higher than we would expect from the amount of luminous matter alone [NEEDSREF]. The existence of this gravitationally interacting matter influences the rate of structure formation in the early universe, and measurements of the cosmic microwave background can state that the observations are consistent with a $\Lambda$CDM model; that is, a model of the universe as composed of non-baryonic, non-relativistic matter [NEEDSREF]. We know very little more about this dark matter other than the conclusions stated above.

At the same time, the nature of physics at high energy (> TeV) scales is unknown. The equations that make up the Standard Model do not extend into this energy range, and current experiments that can produce physics at high energies (the LHC, other colliders) will be limited by the size of the experiments to explore energies of at most 10 TeV. Ther are many theories of physics at high energy scales; all of these predict new particles at higher energies, or additional symmetries. The high-energy particles or new effects at high energies can enter as small corrections in loop diagrams, leading to small corrections in low-energy observables. The electric dipole moment is one such observable that can be measured in laboratories but whose value is sensitive to high energy physics. The most recent measurement of the electric dipole moment of the electron [NEEDSREF] done by the ACME experiment has seen a result consistent with zero, ruling out parameters of 100 TeV theories such as Supersymmetric models. These Supersymmetric (SUSY) theories are high energy theories that solve problems in high energy particle physics, and also predict a stable particle that is a very good candidate for the particle of cold dark matter.

Another probe of this high energy physics is to look for extremely light, feebly interacting particles. These particles arise as the low-mass boson accompanying the breaking of new symmetries at high energies (usually >$10^{14}$ eV). These particles, known generally as Weakly Interacting Sub-eV Particles (WISPs), are good cold dark matter candidates as well, granted that their abundance can be accounted to match the present abundance of dark matter. A particular particle belonging to the WISP category, are axion-like particles (ALPs). This denotes pseudoscalar spin 0 bosons with a non-zero coupling to two photons. ALPs take their name from the axion, a particle with the properties described above which arises from a new symmetry which was postulated by Peccei and Quinn [NEEDSREF] to solve the problem of charge and parity conservation in the theory of strong interactions when a priori there is no physical reason this CP conservation should hold.

As direct detection searches for  the SUSY dark matter candidate with no results, there is an intrinsic neutrino background that will soon limit the experiments' sensitivity within another few orders of magnitude from where they can presently search. In addition, no particles have been seen at the LHC, which expected to find new particles that would be SUSY particles. The EDM experiments have constrained different SUSY models with no observation of the electric dipole moment. At this time, it seems necessary to build small tabletop experiments that can look for other dark matter candidates, such as axion-like particles. There is a wide space in which to search for these WISPs, and having measurements allows us to constrain models. A discovery would give us clues as to the nature of physics at high energy scales, and tell us about the composition of dark matter.

\subsection{The State of Dark Matter Searches}

Starting from the assumption that dark matter is a particle, people look for heavy, stable, neutral particles, known by the acronym Weakly Interacting Massive Particles (WIMPs), under which the SUSY candidate falls. People also look for the WISPs. There are direct and indirect detection searches going on for both candidates. For WIMPS, direct searches look for WIMP to strike a nucleus, causing it to recoil. The indirect searches assume that WIMPs annihilate with each other, being their own antiparticle, and look for signatures of these events in the galaxy.

For WISPs, direct searches primarily use the ALP to two photon vertex, as it is easy to produce large electromagnetic fields in the laboratory and easy to detect photons. Searches assuming that the ALP is the dark matter are only using microwave cavities as resonant detectors [NEEDSREF], although a Rydberg atom detection experiment was performed [NEEDSREF]. These experiments depend on the dark matter density at Earth. There are also new ideas involving the axion as inducing an oscillating electron dipole moment in nucleons [NEEDSREF], which will be sensitive to  ALPs of a different mass than those in the microwave cavity experiments. There are other experiments looking for axion-like particles, although they do not use dark matter as the source of the WISPs. These involve looking for axions produced in the sun (CAST) [NEEDSREF] or directly producing axions from the two photon vertex using a laser and strong magnetic field (ALPS-I, ALPS-II) [NEEDSREF]. The direct production or light-shining-through-wall experiments has the least number of assumptions, but these experiments (solar, microwave cavity, lsw) together form a complementary way of searching for ALPs with the two-photon vertex.

There are also astrophysical arguments that can limit the coupling of the axion to the two photon vertex. The ALP, produced by various processes in red giants, neutron stars, and white dwarfs, would radiate energy from the star, altering the stellar evolution and thus stellar lifetime [NEEDSREF]. By observing stellar lifetime consistent with 10$\%$, constraints on the axion coupling can be derived [NEEDSREF]. A strong limit comes from the observation of neutrinos from the supernova explosion of SN1987A. The observation of neutrinos number and duration consistent with expectations limits the role of axions in acting as a energy loss channel. The limit is on the axion to nucelon coupling, but can be related to the axion to two photon coupling.

This dissertation will focus on a direct search for dark matter axions using microwave cavities as resonant detectors over the frequency range 33.9 to 34.5 GHz.







.
The Standard Model is a highly successful model of particle interactions that can be written as $SU(3)_c \times SU(2)_y \times U(1)$, representing the symmetry groups of the color, hypercharge, and gauge photon symmetries, respectively. The particles and their forces are predicted very well by the framework of the Standard Model, but we believe that there are many mysteries that need an extension of the Standard Model to solve them.

Some of these mysteries are White Dwarf Cooling, Transparency of the Universe to gamma rays, dark matter. The existence of a hidden sector, of particles that do not have electric charge but can interact with Standard Model particles through a mediator, is a general extension to the Standard Model that can be imagined. These also arise in many string theories?

In general we will consider the case for weakly interacting sub-eV particles (WISPs) and possible methods of detecting them.

{\color{blue} next paragraph comes from Patras 2012 paper}

Several theories of particle physics as well as cosmology predict the existence of at least one sub-eV scalar, that is, spin-zero, boson. Correspondingly, many theories of physics beyond the Standard Model (SM) can accommodate scalars with very small masses and feeble couplings to SM fields. An  intriguing possibility in astrophysics and cosmology is that these weakly interacting sub-eV particles (WISPs) may constitute at least some component of the cold dark matter in the universe. It has been shown that these arguments apply for both pseudoscalar, namely axion, and sacalar WISPs, e.g. axion-like particles (ALPs) that couple to two photons.

Many of the extensions of the standard model predict the presence of new particles which can have small masses. For example, the axion and other axion-like particles (ALPs) have been postulated to account for broken symmetries and should behave as weakly interacting sub-eV particles (WISPs). In addition there are hidden sector particles with very low masses that arise from supersymmetry which only rarely interact with standard model particles. The discovery of a new low mass particle  could be a solution to the cosmological dark matter problem.

{\color{blue} Next paragraphs come from Strategy Document}
Weakly INteracting Slim (light) Particles (WISPs) are a feature of many well motivated theoretical models of fundamental physics. one prime example is the axion, a consequence of the Peccei-Quinn solution to the otherwise not explained problem of combined charge and parity conservation in quantum chromodynamics. Moreover, ultraviolet field or string theory embeddings of the SM often include hidden sectors of particles with extremely weak interactions with ordinary matter. The feeble strength of these interactions typically results from underlying new dynamics at energy scales much larger than the electroweak scale. Therefore, probing thsese very weak interactions provides a new window to explore these high-energy scales and will give us crucial cules towards the understanding of the underlying structure of fundamental physics.

At the same time, very weak interactions seem to be a feature exhibited by the two must abundant yet also most mysterious substances in the universe, dark matter and dark energy. Indeed axions and other WISPs are well-motivated dark matter candidates and could even open a door to undertstanding of dark energy. Further motivation arises from anomalous white dwarf cooling and the propagation of very high energy gamma rays in intergalactic space.

\section{The strong CP probelm and the QCD axion}
The combination of charge conjugation (C) and parity (P), CP, is violated in the weak interactions but appears to be conserved in the electromagnetic and strong interactions. The latter constitutes a theoretical puzzle since the most general vacuum of quantum chromodynamics violates CP and on top of that the magnitude of this violation receives a contribution from the the electroweak sector of the SM - which is well known to violate Cp as well. From the experimental point of view, CP violation in quantum chromodynamics would manifest itself in an electric dipole moment of the neutron The absence of any experimental evidence for an electric dipole moment of the neutron at the very strict level of $2.6\times1-^{-26}$ e cm is in strong contradiction to the natural size of the electric dipole moment expected from first principles which is at the $10^{-16}$ e cm level.

Instead of relying on an accidental fine tuning of the CP violation parameters in QCD and weak interactions to explain this phenomenon, Peccei and Quinn pointed out in  1977 that the existence of a new symmetry, spontaneously broken at a high energy scale, denoted by $f_1$ would automatically solve the storng CP problem if it is violated by the color anomaly. The axion is a new pseudo-scalar particle that arises as the pseudo-Nambu-Goldstone boson of this spontaneously broken symmetry. Discovering the axion is thus a way of assessing whether the Peccei Quinn mechanism is realized in nature. For most of the experimental searches, the key axion parameters are the mass $m_a$ and the anomalous coupling to two photons $g_{a\gamma}$. A summary of axion searches based on these two parameters is presented. In this broad parameter space, axion models lie in a diagonal band because the mass and couplings are inversely proportional to $f_a$ and thus directly proportional to each other. This relation holds up to factors of order 1 because different realizations of the Peccei-Quinn mechanism lead to slightly different values of $g_{a\gamma}$.
Originally it was assumed that $f_a$ was close to the weak scale (250 GeV) but collider experiments quickly excluded this possibility, restricting $f_a > 10^5$ GeV. Therefore, the axion must have very small mass, typically sub-eV, and feeble interactions. the axion is therefore a prototype example of a situation where underlying new physics at an extremely high energy scale leads to a new particle that can be probed in high precision low-energy experiments.
\section{WISPs in extensions of the Standard Model}
The case of the axion can be generalized to generic (pseudo-)scalars coupled to two photons, so-called axion-like particles (ALPs). The relation betwen the mass and couplings of the axion is only intrinsic to the color anomaly of the Peccei Quinn symmetry, and thus generic ALPs can show up in all the parameter space.
Like the axion, ALPs can be realized as pseudo-Nambu-Goldstone bosons of symmetries broken at very high energies. Further motivation for their existence comes from string theory .one finds that these theories predict in general a rich spectrum of light (psuedo-)scalars with weakcouplngs. In particular, the compactification of typeIIB string theory, where moduli-stabilization giving rise to ALPs is well understood, can provide the axion plus many ALPs. Of particular interest are so-called intermediate string scales $M_s \sim 10^{10-11}$ GeV, as these can contribute to the natural explanation of several hierarchy problems in the SM. In these theories, ALPs generically exhibit a coupling to photons which lies in a range that is well accessible to some near-future WISP searches.
Hidden sectors, or particles with feeble interactions with ordinary matter, are a generic faeature of field and string theory completions of the SM. For example, a "hidden sector" is commonly employed for supersymmetry breaking. Hidden photons (HPs), gauge photons of an extra U(1) gauge group, are natural ingredients of these hidden sectors. Often, these hidden sectors interact with SM particles only through very heavy particles mediating between both sectors. They thus provide effective couplings - kinetic mixing between photons and HPs - which makes them accessible to experimental searches.In particular, if HPs have a non-zero rest mass, kinetic mixing behaves as mass mixing and thus photon<->HP oscillations occur. This leads to the disappearance and regeneration of photons as they propagate in vacuum. Moreover, the hidden sector naturally can also contain matter with fractional electric charge, often called minicharged particles. Most prominently, they can emerge in theories which contain a hidden photon. MCP searches provide an alternative observational window to the hidden sector and in particular can provide insight if the hidden photon turns out to be massless and thus not directly traceable.
Modifications of gravity proposed to explain the accelerated expansion of the universe often contain new scalar fields. If coupled to matter, they can very easily run into conflict with constraints on new long-range forces. A certain class of these fields evades these bounds by having either masses or couplings or both which depend on the ambient matter density.
\section{Axions and WISPs in astrophysics and cosmology}
\subsection{dark matter}
Progress made in the last decade of cosmological obesrvations has allowed usto establish a standard cosmological model. From the point of view of fundamental physics, its mots remarkable feature is the need for at least two new "substances" not accounted for byt he Sm of particle physics: Dark Energy and Dark Matter (DM). The former can be parameterized by a cosmological constant or some dynamical field and the latter by non-relativistic particles of new species which interact feebly with SM particle and among themselves.
The particle interpretation of DM requires new field beyond the SM. A popular example is the "weakly interacting massive particle" WIMP typically appearing in supersymmetric extensions of th SM. The propsects of exploring the electroweak scale with the Large Hadron Collider has understandably focused the DM seraches on the WIMP paradigm in the last years, and the experimental community has devoted comparatively little effort to explore other possibilities.
However, since the early 1980s axions are know to be very well motivated candidates for DM. Axion cold DM is produced non-thermally in the early universe by the vacuum-realignment mechanism and the decay of topological defects: axion strings and  domain walls. With the axion mass being so small, the phase-space density of cold DM axions is huge and it is conceivable that they form a BEC. These would produce characteristic structure in DM galactic halos.
Axions are not the only WISPs allowing for a solution to DM. The non-thermal production mechanisms attribute to axions are generic to bosonic WIMPs such as ALPS and HPs. A wide range of WISP parameter space can generically contain models with adequate DM density. Finally axions and other WISPs are also produced thermally in collisions of SM particles, contributing to dark radiation or hot Dm.
THe LHC hasn't found anything and neither have WIMP direct searches (my summary)

\subsection{Dark Energy}
\subsection{Hints for WISPs from astrophysics}

\section{HSPs}

I will start with vector bosons that could exist in a hidden sector as the gauge boson of a U(1)h symmetry. This is a simple and general model to start with - these photons are in general the only things that interact with the Standard Model (WHY) and they do it through something called kinetic mixing.

{\color{blue} Next two paragraphs taken from Patras 2013 conference proceedings}
Hidden Sector Photons (HSPs)  are hypothesized particles that arise from adding extra U(1) symmetries to the standard model in a simple extension. They can exist in a wide range of masses and couplings, from $10^{-12}$ eV to $10^6$ eV, with couplings only constrained at the upper limit by tests of the Coulomb law and the magnetic field of Jupiter. At sub-eV masses, the hidden photon's dominant interaction is with the photon via kinetic mixing. The mass difference between the hidden photon and standard model photon allows for oscillations between the two, as in neutrino oscillations or K0 mesons. As these oscillations have not been observed, this suggests that the coupling of the hidden photons to photons is very weak.

There is strong interest in looking for new extensions to the standard model - in a higher mass range (1 MeV) hidden photons could provide the solution for the anomalous muon moment, and in the meV mass range, could provide the explanation for anisotropies in the CMB. More generally, there is scientific interest in searching for extensions to the standard model to explain dark matter and understand the underlying symmetries of nature. Hidden photons, as the particle coming from a U(1) gauge symmetry, are one of the simplest extensions and generic classes of particles one can consider existing. 


\section{Lagrangian: from "A status of the quest for hidden photons by J. Jaeckel"}
" 
THE LAGRANGIAN
\begin{multline}
$$\mathcal{L}_{int} = -\frac{1}{4}W_{\mu\nu}^aW^{a,\mu\nu}-\frac{1}{4}B_{\mu\nu}B^{\mu\nu} \\ - \frac{1}{4}X_{\mu\nu}X^{\mu\nu} - \frac{\chi_Y}{2}B_{\mu\nu}X^{\mu\nu} + \frac{m_X^2}{2}X_{\mu}X^{\mu}+\frac{1}{2}\frac{m_w^2}{g^2}(-gW_{\mu}^3 + g'B_{\mu})^2 \\ +\frac{1}{2}m_W^2(W_{\mu}^1W^{1,\mu}+W_{\mu}^1W^{1,\mu}+\text{ SM matter and Higgs terms}  $$
\end{multline}
where $B_{\mu}$ and $W_{\mu}$ dnote the usual electroweak gauge fileds and $X_{\mu}$ denotes the hidden U(1) filed with gauge coupling $g_X$. Importantly the term $ \frac{\chi_Y}{2}B_{\mu\nu}X^{\mu\nu} $ introduces a mixing between $X_{\mu}$ and $B_{\mu}$.

...stuff about how the mass can come from a Higgs or Stuckelberg mechanism

then next part, "at energies far below the electroweak scale and for small masses of the new gauge boson $m_x << m_W$, we can consider only the remaining light degrees of freedom and the mixing is directly with the photon,
\begin{multline}
$$\mathcal{L} = -\frac{1}{4}F_{\mu\nu}F^{\mu\nu} - \frac{1}{4}X_{\mu\nu}X^{\mu\nu} - \frac{\chi}{2}F_{\mu\nu}X^{\mu\nu}+\frac{m_x^2}{2}X_{\mu}X^{\mu}+j_{\mu}A^{\mu}$$ 
\end{multline}
where the mixing with the photon is related to that with the hypercharge via

$$\chi = \chi_Y \cos(\theta_W)$$"

do a field redefinition

"$$A^{\mu} \rightarrow A^{\mu} - \chi X^{\mu}$$
$$X_{\mu} \rightarrow X^{\mu}-\chi A^{\mu}$$

The second field definition allows you to write the lagrangian as

\begin{multline}
$$\mathcal{L} = -\frac{1}{4}F_{\mu\nu}F^{\mu\nu}-\frac{1}{4}X_{\mu\nu}X^{\mu\nu} +\frac{m_X^2}{2}(X_{\mu}X^{\mu}-2\chi X_{\mu}A^{\mu}+\\ \chi^2 A_{\mu}A^{\mu}) + j_{\mu}A^{\mu}$$
\end{multline}

...non-diagonal mass term mixing X and A

Similar to neutrino flavor oscillations

\section{Techniques for Detection}

The way to detect HSPs terrestrially is through an LSW technique. I don’t know of any other, at least at low masses. At higher masses, you can look at accelerators for missing momentum or energy in scattering experiments. Astrophysically you can look for energy loss in stars (this applies to HSPs right?) and thus set limits on their coupling and mass. The most stringent limits are set through the Sun.

\section{LSW technique}

The LSW technique looks for energy transmission between two isolated volumes through the mediation of a hidden photon. We use microwave light to set limits of hidden photons with mass 140 microeV. We also use resonant cavities to enhance the sensitivity by a factor of each cavity quality factor.

\section{Switching to ALPs}

ALPs, or axion-like particles, also form an important family of possibile particles to search for. To explain their relevance, we will have to start with the case for the axion. Axion-like particles are generalizations of the axion, as we will see later, that do not necessarily solve the particle physics symmetry puzzle that the axion does, but have the same properties to be cosmologically significant and interesting to search for in their own right.

\section{Axions as Solution to the Strong CP Problem}

The strong CP problem is a statement of a puzzle involving the non-observance of CP violation in the strong sector. This means that in the QCD Lagrangian, there is a term that violates CP; however, we do not observe CP violation (as constrained by the neutron EDM) which constrains the parameter of this term, $\bar{theta}$, to less than one part in $10^{10}$. It is remarkable that this term is so close to zero when one considers that it is the sum of two separate and presumably independent parts - a $\theta$ parameter that arises from QCD considerations of the non-trivial vacuum ($\theta$ represents the density of instantons) and the phase of the CKM matrix, which is an electroweak sector thing. The solution to the strong CP problem is generally considered to be the axion, although there are some other existing solutions (Nelson-Barr model, weak CP breaking). The axion field, as proposed by Peccei and Quinn



\section{ CP violation and teh strong CP Problem}

The axion solves the strong CP Problem.
The strong CP problem is tied to the $U(1)_a$ problem.
The U(1)a problem relates to the $\eta$ not having the right mass.
If you introduce an axial current, the Adler-Bell-Jackiw anomaly results.
This gives the right masses, but also introduces a CP violating term.
CP violation means that $x \rightarrow -x$ and $c \rightarrow -c$ do not change the equations of motion.
CP violation was observed in the electroweak sector by Cronin and Fitch in the 1960s. The neutral Kaon is different from its anitparticle; however the eigenstates of the system are not CP egienstates, so CP is not a good symmetry.
The CP violating term in the QCD lagrangian has two components. One is theta, the other arg det M. The CKM matrix is symmetric but non-Unitary? There is something about the number of quark families being N=3 that makes it so that the matrix has CP violation.
$\theta$ is described as the density of instantons.
The combination of $\theta$ and arg det M is $\bar{theta}$, observable.
The neutron edm should be an observable consequence of theta bar being nonzero.
However it is less than $10^{-10}$.
To see that the neutron EDM is CP violating, notice that the EDM transforms oppositely to the MDM. 

\section{THe PQWW axion}

\begin{figure}
\includegraphics[scale=0.7]{RogersAxionInteractions}
\end{figure}

PQ introduced a new global chiral U(1)pq symmetry \cite{pecceiquinn}. THe potential would have spontaneous symmetry breaking and explicit symmetry breaking. The explicit symmetry breaking is due to instanton effects in QCD. The Mexican Hat potential tilts and the curve of the potential is the symmetry breaking scale $f_a$. The value of $\bar{theta}$ tends towards zero which is CP conserving and solves the strong CP problem. Weinberg and Wilczek \cite{weinberg},\cite{wilczek} noticed that when the spontaneous symmetry breaking first appears, the angular degree of freedom of $\bar{theta}$ is not fixed - this degree of freedom is the axion. When the potential tips due to instanton effects the axion acquires a mass, as previously mentioned.

The potential was first posutlated to be made from two Higgs doublets. Their value $\sqrt{\lambda_1^2 + \lambda_2^2 }$ would be the symmetry breaking scale $f_a$. For Higgs doublets somehow this relates to the electroweak scale. THis would give a mass of the axion of around 150 keV. This would be observable in accelators; particular in decays of K+ \cite{kdecays}, J/Psi \cite{jpsidecays}, and Y \cite{upsilondecays}. None of these were boserved. The principal experiments were at SLAC, such as the Crystal Ball experiment.

New models for the axion were created that removed the dependence on the electroweak scale by adding high energy extensions. The KSVZ model \cite{ksvz}, named after Kim, Shifman, Vainshtein, and Zhitnikskhii, introduces an exotic heavy quark and high energy scalar field. There are no PQ charges. The DFSZ model \cite{dfsz} introduces two new higgs doublets but adds PQ charges to the fermions. THis allows for tree-level couplings of electrons to axions. THe KSVZ model is known as the hadronic model because it does not allow for tree-level couplings of electrons to axions. However, axions couple to gluons in all the models, and axion coupling to photons also appears as a consequence of the gluon axion coupling, and is thus also generic. 

The new models allowed for the existence of axions with much higher symmetry breaking scales. Since the mass and coupling of the axion are inversely proportional to f, this meant much lighter axions with weaker interactions. Thus the axion of these models was dubbed the “invisible” axion. However, Pierre Sikivie \cite{sikivie} proposed using haloscopes to look for galactic axions and helioscopes to look for solar axions, both using the axion to photon coupling.

The haloscope idea relies on the assumption that axions form part of the galactic dark matter. We will now discuss why axions make good dark matter candidates.

\section{Axions as Dark Matter Candidates}

Axions, if they exist, are light and weakly interacting. They have no electric charge, and thus by virture of their feeble interactions with ordinary matter, seem to be a candidate for dark matter. Furthermore, if axions are produced non-thermally, they can exist as a non-relativistic gas of particles, allowing them to fall into the category of “cold dark matter”, which is observationally favored as the type of dark matter that would allow the galactic structure we see today “bottom-up”.

The proposal for axion generation in a non-thermal way comes from what is known as the misalignment mechanism. This refers to the fact that when the axion acquires mass due to the tipping of the Mexican Hat potential (due to instanton effects at the QCD phase transition) the initial value of $\bar{theta_i}$ is not 0. The oscillations of $\bar{theta}$ around its minimum allow for the axion to have … does this have to do with Hubble expansion? Something something and this allows for the axion to be produced as a zero-momentum gas. The axion will then virialzie and thus have an energy dispersion equal to the virial velocity dispersion of the Milky Way $10^{-3}$. 

There exists another way for the axion to be produced, which depends on at what temperature inflation occurs, or perhaps more accurately, when reheating or thermalization occurs at the end of inflation. If $T_{reheat} < T_{pq}$, the axion correlation length will only be causally connected in small patches of the sky. At poitns where two unconnected regions meet, there will be an axionic string - what I mean here is a 1-D topological defect caused by the breaking of teh axial PQ symmetry. The domain walls caused by these strings can radiate and make axions. There is a “domain wall problem” which may or may not be an issue. If $T_{reheat} > T_{pq}$ then it is as if inflation doesn’t matter -we will expreience a spatially homogeneous value of the $theta_{bar}$ and the only contribution to the axion density will come from the misalignment mechanism. FOr some reason thermal axions don’t matter. Perhaps the expansion of the universe dilutes them or something.

\section{Astrophysical and Cosmological Bounds}
The existence of axion dark matter has cosmological consequences - for one their density cannot exceed the critical density, which would lead to an “overclosure bound” on the number of axions (which ends up being a lower bound on the mass). However, this limit is subject to some controversy as it depends on whether the axion density is influenced by inflation or not. There might be other reasons to doubt it to. 

The existence of axions can also have observable effects in astrophysics. As weakly interacting particles, if produced in the core of some stellar object, axions can freely stream away, carrying some of the object’s energy with them and thus hastening energy loss. The primary process by which this occurs is the Primakoff process, where two photons can produce an axion, the two photons being provided by the strong electromagnetic fields in plasmas that occur in the Sun. In white dwarfs, the primary mechanism is bremsstrahlung; in red giants and HB stars, it is Compton-like processes. In all of these energy loss arguments, the lifetime of these astrophysical objects is observed to be within $5-10\%$ of theoretical expectations, placing an upper bound on axion couplings. In particular, horizontal branch stars’ non-observation of lifetime shortening leads to a limit of $g_{a\gamma} < 10^{-10} GeV^{-1}$. 

\begin{figure}
\includegraphics[scale=0.7]{RogersAxionEmissionProcesses}
\end{figure}

Putting it all together, axions are generally speaking, restricted to lie within the 1 microeV to 1meV range. The coupling has been also constrained to be less than $10^{-10} GeV^{-1}$. However, there are people searching in other mass ranges and for couplings weaker than those implied by the DFSZ and KSVZ axions. The reason is taht the axion production mechanism could be generically applied to any particle that undergoes spontaneous symmetry breaking and has an anomaly with some other symmetry, not necessarily QCD. The only difference between this particle and the QCD axion would be that the mass of the axion-like-particle ALP would not be tied to the coupling constant, so they would be two free parameters. 


\section{Solar and Laboratory Bounds}

Within these astrophysical and cosmological constraints, direct searches have taken place for axions converting to photons in terrestrial experiments, using either external sources of axions (from the sun or as dark matter) or producing them directly in the lab via the Primakoff effect.

The helioscope experiments have their best bound with the CERN Axion Solar Telescope (CAST) which to-date has excluded $g_{a\gamma} < 8 * 10^{-11} GeV^{-1}$ for $m_a < 0.02$ eV. The haloscope experiments are dominated by microwave cavity experiments. In the 1980s, RBF and UF excluded … … 16 microeV. ADMX has now excluded 1.9 to 2.6 microeV. Experiments using selective ionizatio of Rydberg atoms have also placed limits in the 10-50 microeV range. Further upgrades to ADMX such as ADMX-HF are in place to cover the range to 80 microeV in the next four years.

\begin{figure}
\includegraphics[scale=0.7]{g_gamma_param_space}
\end{figure}

We will be using a microwave cavity experiment in this work to place limits on the coupling for masses of 140 microeV.

{\color{blue} REDUNDANT MATERIAL BELOW}

\section{Redundant Material}
\subsection{Axion Cosmology}

Axion generation comes from particle physics concepts. The main problem is called the strong CP-problem, which expresses the puzzle that charge parity symmetry is conserved very very well by the strong interaction even though we have seen that it is not a universal symmetry due to the decay of stuff in the weak force I think it is K mesons.

One solution to the strong CP problem is to introduce a new field with one free parameter. This makes it a $U(1)$ field. This field must be chiral, whatever that means. This field will "dynamically" cause any CP violating terms ot go to zero. Basically I have no idea what this means. At some point I should talk about the nature of the QCD vacuum. It is special and has to do with instantons.

\subsection{Adler-Bell-Jakiw Anomaly}

The field will symmetry break for some reason, and cause a particle. This is the axion. The axion will have a mass inversely proportional to its couplings with matter. This makes it a well-defined particle to search for, since for a given mass, we know what sensitivity we need in order to detect it.

The axion was looked for but then wasn't found for 1keV masses. Look this up. Then people had to make up new higgs and fields to make the axion work. This is known as the FDSV and KDSM model. The evolution of the axion in time has something to do with inflation.

Talk about axion domain walls and strings?

Either the axion is born through SSB or through a mexican hat potential sloshing around and QCD vacuum topology or through oscillations of the field around the expectation value at zero. 

Something about the Friedmann equations

{\color{red} Notes Below}

\section{Notes from Arias paper -Review of Dark Matter}
\subsection{Dark Matter}

From WISPy Cold Dark Matter (P. Arias 2012)

Dark matter must have weak intreactions with standard model particles and be cosmologically stable. It must also be cold to allow for structure formation.

WIMPs (TeV scale) would be thermally produced; their large mass means they would be non-relativisitic, i.e. cold. The interactions are small because the mediator particles are heavy (W and Z). However their large mass means they can decay to many things, so they wouldn't be cosmologically stable. Therefore symmetries must be introduced to make them so. Two examples are the lightest SUSY particle with R-parity, or the lightest Kaluza-Klein model with conserved parity in extra dimensions.
You can achieve stability by having a smaller mass particles. However if thermally produced, the light particle may not be cold enough. To say it another way, the free-streaming length would increase with smaller mass so structures would not form. This is how standard neutrinos are ruled out as DM candidates. You can also rule out eV mass axions.
You can make light DM particles without thermal production. ALPs appear generically in string compactifications. Once you make this light DM stuff, it's hard to reabsorb it. There is a huge allowed parameter space.
\subsection{Misalignment Mechanism}
It assumes the fields in the early universe have a random initial state. These may arise from quantum fluctuations during inflation. They are fixed by the expansion of the universe. The evolution of the fields depends inversely to their mass $t \sim m^{-1}$. After that time t, the fields attempt to minimize their potential, producing oscillations. Damping comes via decays, but if there are no significant decays, the oscillations behave as a cold dark matter fluid. This is because their energy density is diluted by the expansion of the universe $\rho \propto a^{-3}$.
For a real scalar field with a Lagrangian
$$L = \frac{1}{2}\partial_u \phi \partial^u \phi - \frac{1}{2}m_{\phi}^2\phi^2 + L_I$$
where $L_I$ is the lagrangian part for the interactions of the scalar field with itself and everything else. Assume the universe undergoes a period of inflation at a value of the Hubble expansion parameter $H = \frac{d log a}{dt}$ larger than the mass. After inflation the field is approximately spatially uniform and the initial state has one single value $\phi_i$. After inflation a period of reheating occurs, and afterward radiation dominated expansion. The equation of motion for $\phi$ in an expanding universe is 
$$\\ddot phi + 3 H \dot \phi + m_\phi^2 \phi = 0$$
The mass is a function of time and is dependent on $L_i$.

There are regimes of the solution. When $3H \gg m_\phi$, $\phi$ is an over damped oscillator and gets frozen $\dot \phi = 0$. When at $t_1$ you get $3H(t_1) = m_\phi(t_1) \equiv m_1$, it is underdamped and the field oscillates. The mass term is the leading term in the equation and the WKB approximation gives the solution
\[
\phi \simeq \phi_1 (\frac{m_1 a_1^3}{m_\phi a^3})^{1/2} \cos{(\int_{t_1}^t m_\phi dt)} 
\]

where $\phi \sim \phi_i$ since up to $t_1$ there is no evolution. This is valid for all phases (radiation dominated, energy and matter dominated) of the universe.

This solution has fast oscillations with slow amplitude decay. The amplitude is the pre-factor and the phase $\alpha(t)$ the integral. The energy density of the scalar field is 
\[
\rho_\phi =\frac{1}{2}\dot \phi^2 + \frac{1}{2}m_\phi^2 \phi^2 = \frac{1}{2}m_\phi^2 A^2 + \cdots \]

where the dots are the derivative of A terms and we assume them much smaller than $m_\phi$.  The pressure is 
\[ 
p_\phi = \frac{1}{2} \dot \phi^2 - \frac{1}{2} m_\phi^2 \phi^2 = -\frac{1}{2} m_\phi^2 A^2 \cos{2\alpha} - A \dot A \sin{2\alpha} + \dot A^2 \cos^2{\alpha} 
\]

At times $t \gg t_1$, the oscillations occur at time scales $1/m_\phi$ much much master than the cosmological evolution.Then we can average over the oscillations:
\[ 
\langle p_\phi \rangle = \langle \dot A^2 \cos^2{\alpha} \rangle = \frac{1}{2} \dot A^2 \]

Since $\dot A \ll m_\phi A$, to leading order, the equation of state is 
\[
w = \frac{ \langle p \rangle}{\langle \rho \rangle} \simeq 0 \]
which is exactly that of non-relativistic matter.

The energy density in a comoving volume $\rho a^3$ is not conserved if the mass changes in time.

However $N = \rho a^3/m_\phi = \frac{1}{2} m_1 a_1^3 \phi_1^2$ is constant; think of it as the comoving number of non-relativistic quanta of mass $m_{\phi}$. Then the  energy density today is
\[
\rho_{\phi}(t_0) = m_0 \frac{N}{a_0^3} \simeq \frac{1}{2}m_0 m_1 \phi_1^2 (\frac{a_1}{a_0})^3
\]

Where 0 is present time.
If you use temperature instead of times and scale factors, it is clearer. First, note the conservation of comoving entropy: $S = sa^3 = 2\pi g_{*S}(T)T^3a^3/45$. Thus you can write $(a_1/a)^3 = g_{*S}(T)T^3/g_{*S}(T_1)T_1^3$. The expression for the Hubble constant in the radiation dominated era is $H=1.66\sqrt{g_{*}(T)}T^2/m_{PL}$ and the definition of $T_1$ is $3H(T_1) = m_1$. The fundtions $g_*$ and $g_{*S}$ are the effective numbers of energy and entropy degrees of freedom. Then the dark matter density today can be expressed as
\[
\rho_{\phi,0} \simeq 0.17 \frac{keV}{cm^3} \times \sqrt{\frac{m_0}{eV}}\sqrt{\frac{m_0}{m_1}}(\frac{\phi}{10^11 GeV})^2 F(T_1)
\]
where $F(T_1) \equiv (g_*(T_1)/3.36)^{3/4}(g_{*S}(T_1)/3.91)^{-1}$ ranges between 1 and 0.3. The abundance depends mostly on the initial amplitude, and to a lesser extent on todays mass. The factor $\propto 1/\sqrt{m_1}$ reflects the damping of the oscillations in the expanding universe. If $T_1$ is smaller, $m_1$ and $H_1$ are smaller, so it will be less damped.

THe DM density measured by WMAP and other large scale structure probes is
\[
\rho_{CDM} = 1.17(6)\frac{keV}{cm^3}
\]
so we need really large $\phi_1$ or small $\phi_1$ and small $m_1$.

If you want the condensate to have the behavior of standard cold dark matter, then at latest when the matter and radiation equaled each other $T_{eq} \sim 1.3$ eV, the mass should have its current value $m_0$ so that the DM starts to scale truly as $1/a^3$. By this point the field should have already started to oscillate. These constraints would limit $m_1 > 3H(T_{eq}) = 1.8 \times 10^{-27} eV$, so there is a limit on $\rho_{\phi,0}$:
\[
\rho_{\phi,0} < 1.17 \frac{keV}{cm^3} \times \frac{m_0}{PeV} (\frac{\phi_1}{53 TeV})^2
\]
When particles are produced by the misalignment mechanism, they are semi-relativistic. Their momenta are on the order of the Hubble constant $p \sim H_1 \ll T_1$, so the velocity distribution has a very narrow width of roughly
\[
\delta v(t) \sim \frac{H_1}{m_1} (\frac{a_1}{a_0}) \ll 1
\]
Combined wit hthe high number density of particles, $n_{\phi,0} = N/a_0^3 = \rho_{CDM}/m_0$, this narrow distribution typically leads to very high occupation numbers for each quantum state,
\begin{multline}
N_{occup} \sim \frac{(2\pi)^3}{4\pi/3}\frac{n_{\phi,0}}{m_{0}^3\delta v^3} \sim 10^{42} (\frac{m_1}{m_0})^{3/2}(\frac{eV}{m_0})^{5/2}
\end{multline}
where we have used $a_0/a_1 \sim T_1/T_0 \sim \sqrt{m_1 m_{PL}}/T_0$. If the interactions are strong enough, thermalisation occurs, and a BEC can form.

\subsection{Axion-LIke Particles}
Meaning particles with only derivative couplings to matter, and in particular an interaction with photons given by $L = -\frac{1}{4}g\phi F_{\mu\nu}\tilde F^{\mu\nu}$. g has dimensions.

Cosmology depends on the type of interaction generating its mass and in particular how this mass changes though the evolution of the universe.
\subsubsection{ALPs from pNGBs and string theory}

When a continuous global symmetry is spontaneously broken, massless particles appear in the low energy theory: Nambu Goldstone bosons. They appear in the Lagrangian as phases of the high energy degrees of freedom. Since phases are dimensionless the canonically normalized theory at low energies always involves the combination $\phi/f_{\phi}$, where $\phi$ is the NGB filed and $f_{\phi}$ is a scale close to the SSB scale. The range for $\phi/f_{\phi}$ is $(-\pi, \pi)$ so $phi_1 \sim f_\phi$. String axions however appear in all compactifications. They also have a shift symmetry and are periodic but $f_\phi$ is the scale of the string length�it gets complicated.
All of the global symmetries in the standard model are broken (assumption: neutrinos Majorana fermions, otherwise B-L an exception). The black hole no hair theorem and what we know about quantum gravity tell us this should ultimately occur to any additional global symmetries. So we should have pseudo NGBs. They then have a mass.
You can break the shift symmetry explicitly, spontaneously, perturbatively, or non-perturbatively; for stringy axions, the shift symmetry is exact to all orders of perturbation theory and is only broken non-perturbatively, either from a non-abelian anomaly, gauging condensation or stringy instantons. The ALP potential can be written as
\[
V(\phi) = m_\phi^2 f_\phi^2 (1-\cos{\frac{\phi}{f_\phi}})
\]
The mass of the ALP is independent of the temperature unless generated by a sector that is thermalised. As long as $\phi/f_{\phi}$ small, it satisfies the equation of motion above�some inaccuracy can be corrected except in border cases.

g can be written as 
\[
g \equiv \frac{\alpha}{2\pi}\frac{1}{f_\phi}N
\]

N can be an integer or more complicated if ALP mixes with other ALPs/pseudoscalar mesons.

If we call the initial misalignment angle $\theta_1 = \abs{\phi_1}/f_{\phi}$, then $\phi_1 = \theta_1 \frac{\alpha N}{2 \pi g}$

Discussion of preferred masses, don't get it.

\subsubsection{ALP dark matter from misalignment mechanism}

$\phi_1$ depends on the behavior of the ALP filed during inflation. For a pNGB, the SSB could take place before/after inflation: the pNGB exists effectively only after SSB and it is during the associated phase transition that its initial values are set.?? For a string axion with certain parameters, the ALP filed will take random values in different causally disconnected regions of the universe. The size of the domains can't be bigger than
\[
L \sim \frac{1}{H_{SSB}} \sim \frac{m_{PL}}{f_\phi^2\sqrt{g_*(f_\phi)}}
\]

If SSB happens after inflation, the DM density has inhomogeneities of order 1. Non-linear effects due to attractive self-interactions caused by higher order terms in the expansion of the potential drive the overabundances to form peculiar DM clumps - mini clusters. Minicluster mass leads to structure formation. MInicluster mass set by dark matter mass inside Hubble horizon $d_H = H^{-1}$ when self-interaction freezes out. $M_{mc} \sim \rho_{\phi}(T_\lambda)d_H(T_\lambda)^3$ for freeze out temp $T_\lambda$. Long range interactions exponentially suppressed for distances longer than $1/m_\phi$, so $T_\lambda$ on the order of $T_1$. For QCD axions this is true. For QCD axions $M_{mc} \sim 10^{-12} M_{sun}$. There is a low lower bound by CDM power spectrum.
You can observe mini clusters that survived tidal disruption during structure formation with lensing experiments. �
During SSB topological defects and domain walls can form. Strings have a thickness and typical sizes of xx. They can reconnect, form loops and decay into pNGBs. Long discussion about axion emission spectrum, domain walls, ends in reference to Pierre Sikivie's lecture notes on axion cosmology. Everything said for QCD axions applies to ALPs.

\subsubsection{Sufficient production}
General constraint: need enough DM and also that mass at matter radiation equality be greater than Hubble constant. ??
For pNGB ALPs also have that the field value itself cannot be larger than $\pi f_\phi$. This gives you the limits on the possible regions.

DARK MATTER ALP M-G PLOT
\begin{figure}
\includegraphics[scale=0.7]{mgplot}
\end{figure}

You can make stronger bounds with more assumptions. One model could be that $m_\phi$ is constant throughout the universe expansion. This is the bound on the bottom, with N assumed to be 1 and $theta_1$ tuned to give the right DM abundance.

Something about scale of quantum fluctuations during inflation placing lower bound on $\phi_1$.

ALP is effectively massless during inflation. So intomogeneities = isocurvature perturbations of the gravitational potential. WMAP7 sets very stringent constraints on isocurvature perturbations
�complicated.


If ALP acquires a mass due to coupling with a hidden non-abelian group, then the associated global symmetry is anomalous, just as the $eta'$ acquires its mass from QCD instantons. For our ALP we need another unbroken SU(N) group, which condenses at a scale $\Lambda$. Then
\[
m_\phi \simeq \Lambda'^2/f_\phi \equiv m_0
\]
for $T \ll \Lambda$

or 
\[ m_\phi \simeq m_0 (\frac{\Lambda''}{T})^{\beta} 
\]
for $T \gg \Lambda$.

T is the temperature of the new sector. All the lambdas and the primed lambdas are roughly the same. At temps greater than $\Lambda$, electric screening damps long range correlations in the plasma and thus the instantoic configurations, resulting in a decrease of the ALP mass. That's how you get the exponent $\beta$.

Discussion follows. I don't get it.

\subsubsection{Survival of the condensate}

The ALP CDM scenario can be tested via its coupling to photons. This could be bad because it gives the ALPs a decay channel $\phi \rightarrow \gamma\gamma$. The corresponding lifetime in vacuum is
\[
\tau_\phi \equiv \frac{1}{\Gamma_{\gamma\gamma}} = \frac{64 \pi}{g^2 m_\phi^3}
\]
it keeps on going but I stopped writing. ALPS with a lifetime shorter than the age of the universe have to be discarded.

Secondly ALPs from the condensate can be absorbed by a thermal photon which is either on-shell ($\gamma \phi \rightarrow \gamma^*$) or off-shell ($\gamma^*\phi \rightarrow \gamma$). Off-shell photons are understood to be absorbed or emitted by another participating particles, with for example in the inverse primakoff process the extra particle being a charged particle from the plasma.

THer thermalization rate of ALPs due to the Primakoff process was found to be something. This is much faster than the decay rate. Long discussion I don't understand. Something about primordial magnetic fields.

\subsubsection{thermal population of ALPs}

Primakoff process can make a thermal ALP population. This has been reviewed. It is mostly excluded unless you wiggle out of it.

\subsubsection{Detecting Photons from ALP decay}

Even if the lifetime is longer than the age of the universe some ALP decays happen and the resulting monochromatic photons can signal the existence of ALP CDM.

Although the isotropic model turns out not to be correct (see Hotz's thesis), people use it anyway.

What if our lab is at a zero node of the axions could this happen? Do we care about caustics?

\section{Miscellaneous Text}

coherence length. The de Broglie length of the axion is around 1 m for our .14 meV axions. Check this.

I can derive the equations of motion expected and why the interaction is $E\cdot B$ for pseudoscalars and $B\cdot B$ for scalars. I couldn't really explain it in terms of conservation of symmetry properties though.

For the velocity dispersion we assume a Maxwellian distribution. Not sure exactly what this means. Also something about the terminal escape velocity from the earth.


The coupling factor depends on the mass of the up and down quark and the N, the number of species. The KFSV and DFSV models only add a prefactor of order 1 to it.

A. Ringwald (Seattle axion workshop) says that there are other models that could account for axions, lower in mass that $10^{-6}$ eV.


\subsection{Hidden Photon Cosmology}

At higher energies I know people are looking for hidden photons (called dark photons now) to explain the magnetic moment of the muon via some kind of bremstrahlung. Perhaps other people look for them at even higher energies.

As far as I understand it the hidden sector photon comes by way of string theory or just saying there is this extra U(1) field with mass.

Some people think they could also be dark matter candidates. In Ann Nelson's review there is the Higgs mechanism or Stuckelberg mechanism.

\subsection{ALPs}

Other two cavity experiments are ALPs, GammeV, and Chase. ALPs at Desy uses extremely long optical guides in a magnetic field with a barrier between them. They do not use resonant cavities. CammeV did use resonant cavities.
These experiments are important because they do not rely on dark matter models or assumptions that axions are dark matter.
CAST assumes axions come from the sun. This is a very broadband source of axions and they are able to make very good limits.
Other proposals for axion detection involve looking at the bending of two lasers. Aaron Chou.
I should also mention PVLAS. They looked for a change in polarization due to axions leaving a laser field leading to birefringence. They later retracted their results as they were non reproducible. Later experiments (GammeV) did not see any effects for the same detection sensitivity.
Birefringence experiments are being done in Grenoble I think.

\subsection{Outlook}

The Rydberg atoms experiment CARRACK was the closest that came to the concept of what I'm trying to do, masers.

\section{Cavity Design}

Discussion comes from Cosmic Axion Workshop book "Experience With Florida Axion Detector" section
"""
Because the axion frequency is unknown one needs to address the problem of constructing cavities over a wide range of frequencies while maintaining a large volume and Q. The easiest way to go to higher frequencies would be to use higher modes of the same cavity. The trouble with this is that the form factor C is a rapidly decreasing function of mode number $C \propto f^{-2}$. In the anomalous skin depth regime $Q \propto f^{1/3}$ and so ...

There are two major difficulties with the use of metallic or dielectric inserts, resonance mode crossing and field localization. The former refers to the fact that the mode density of TE modes near the lowest TM mode increase like $f^2$. In the vicinity of each crossing modes are getting mixed and repel each other which leads to holes in the scanned frequency space. Furthermore Q is degraded due to mixing.

Another important issue is tuning of the cavity. The principal way to do this is by displacing a dielectric or metal rod inside the cavity. PUshing a rod into the cavity is a possible mechanism but it can lead to longitudinal localization as can be seen... to avoid the loss in C one has to use a thin rod which reduces the tuning range. 
"""

from "Experience with the KEK detector on Galactic Axion Research"
""
Most theoriest s require some kind of viable axions to explain the CP non-violation in the strong interaction if they are unwilling to admit a massless quark. Exceptional is a paper by Pakvasa and Sugawara who pointed out the possible mechanism which avoids the presence of axions without having a massless quark. Furthermore recently there appeared other solutions related with a wormhole for strong CP problems

In order to explain some astrophysical phenomena, especially the flat rotation curve, the movement of binary glaxies or the abundance of nuclear elements, the existence of non-luminous mass as much as the factor of 10 to 100 of the luminous mass is required. The candidates could be any particles or even black holes, if they are non-massless ie gravitationally interacting and non-gravitationallly non-interactive or weakly interactive and vialbe during the evolution of the universe. The neutrinos are the most promising as the existence is actually confirmed. The upper limit of the electron neutrino is less than 11 eV through the neutrino burst observation of SN1987A. However..the neutrinos are hot, that is, relativistic, which hinders the growth of fluctuations in a galactic scale. Some theories and computer simulations favor a cold dark matter, at least partially, to expalin the formation of galaxies and clusters of galaxies.

We consider interactions where an axion interacts with two quarks directly, and each of the quarks couples to a photon. At low energy overall reactions can be formulated into a single event a->2gamma. The scheme is the same as the neutral pion decay into two photons, and iti s possible for an axion to couple with a photon and decay into another photon via the Primakoff effect...

An axion current $j_{\mu}^a$, definied by the Peccei-Quinn current $j_{\mu}^{PQ}$ minus a linear combination of the chiral currents $\bar{u}\gamma_{\mu}\gamma_5 u$ and $\bar{d}\gamma_{\mu}\gamma_5 d$
\[

j_{\mu}^a = j_{\mu}^{PQ} - \frac{N}{1+z}(\bar{u}\gamma_{\mu}\gamma_5 u + z \bar{d}\gamma_{\mu}\gamma_5 d)
\]

leads to
\[

\partial^{\mu} j_{\mu}^a = \frac{e^2}{16\pi^2} N(\frac{8}{3}-\frac{2}{3}\frac{4+z}{1+z})F_{\mu\nu}\tilde F^{\mu\nu} -  \frac{N}{1+z} 2i(m_u\bar{u}\gamma_5 u + zm_d \bar{d}\gamma_5 d)

\]

where z will be fixed by Dashen's formula. Ni s the trace over all Weyl fermions of the Peccei-Quinn charge $Q^{PQ}$ and the color charge $Q_{\text{color}}^{\alpha}$, any one of the eight generators of$SU_c(3)$

\[

N = Tr[Q^{PQ}(Q_{\text{color}}^\alpha)^2]

\]

Meanwhile a pion current 

\[

j_\mu^3 = \frac{1}{2}(\bar{u}\gamma_{\mu}\gamma_5 u + \bar{d}\gamma_{\mu}\gamma_5 d)

\]

Requiring that the axion-pion mixing terms in Dashen's formula vianish, one finds that $z=m_u/m_d$ and the axion mass $m_a = \frac{m_\pi f_\pi}{v}N\frac{\sqrt{z}}{1+z}$

Krauss et al describe the galactic axion field by Fourier representation
\[

a(\omega) = (2T)^{\int_{-T}^{T} a(t)e^{i\omega t} dt

\]

the magnitude of the axion power spectrum is determined by the requirement that 

\[

<\rho>_{\text{halo}} = m^2<a^2>

\]

For a self-gravitational isothermal sphere of particles that have been thermalized via violent relaxation, a Maxwellian velocity distribution is expected.

Cylindrical Cavity

The rf contact and vacuum seal is ensured by an indium wire....the quality value at room temperature is xx% of the theoretical  value. A sapphire rod is used as a tuner by moving along the cavity axis. At an extreme case where a dielectric rod is put along the central axis of a cavity both ends touching the bases, the frequency fshift is given by xx

or a tuning range of 30%
"""

from "Design for a practical laboratory detector for solar axions"
"""

"""

from "An Experiment to Produce Light Pseudoscalars and QED vacuum polarization"
"""
Many current theories predict the existene of pseudo Nambu-goldstone bosons. These particles would have very small mass and unkown extremely weak coupling to fermions. They are in general pseudoscalars (axions) but can also be scalars (majorons). SInce they are not massless, such bosons couple to two photons through a triangle anomaly graph. The coupling to $\gamma\gamma$ is
\[

g_{\alpha\gamma\gamma} = \frac{g}{4\pi^2 m}e^2 = \frac{a}{\pif_\alpha}

\]

where g is the Yukawa coupling, and m is the mass of the particles in the fermion loop; instead, in analogy to $\pi^0$ decay, we introduce the empirical decay constant $f_\alpha$. In axion models $f_\alpha$ is realted to the pion decay constant through 
\[
m_\alpha f_\alpha = m_\pi f_\pi$

The coupling g (or $g_{\alpha\gamma\gamma}$ must be weak to avoid contradiction with the absence of such particles in several production and decay experiemnts. Limits on $g_{\alpha\gamma\gamma}$ can also be obtained from the cooling rate of stars and indicate that $g_{\alpha\gamma\gamma} < 10^{-8} GeV^{-1}$. Because of their weak coupling axions have been considered as candidates for the dark matter of the universe. In that case the axion mass is predicted to be $m_\alpha \approx 10^{-5} eV$ and the axions would have condensed into galaxies. The density is sufficient to make possible the detection of these cosmic axions if their coupling $g_{\alpha\gamma\gamma} \ge 10^{-14} GeV^{-1}$. An experiment searching for cosmic axions in the mass range $4.5\times10^{-6} < m_{\alpha} < 3 \times 10^{-5}$ eV is in progress at Brookhaven. Such experiments are based on the assumption that the axions are cold and thus converted, give rise to a narrow line the microwave spectrum of the detector.

In contrast the production of axions in the laboratory - if it can be detected - is not dependent on the existence of a halo of cold axions and does not require a prior knowledge of the axion mass. We are running an experiment that would detect the production of axions (or scalars) if ... numbers. While limit is not as low as that obtained by the cosmic axion search, it explores a broad region of axion masses, at a level of couplings for which no experimental information exists. Even though theoretical models cannot predict the exact mass of the pseudoscalars (or scalars) in this region, the theoretical motivation for such a search is quite strongat this time.

Axion

The QCD lagrangian contains a term xx where xx is the gluon field, $\alpha_s$ is the strong running coupling constant, and $\theta$ is an angular parameter.

This term violates CP symmetry (C: charge symmetry, P: parity symmetry) for $\theta \neq 0$ or $\theta \neq \pi$. At the time the theorists put it in the Lagrangian, it was most appropriate to describe the attitude prevalent at the time, stemming from the fact CP violation had been seen in the weak interactions, whose origin by the way is not yet known.

There is a series of vacua, distinguished by a topological number n, which are the classical solutions of the gauge equations in QCD but are not invariant under all possible gauge transformations. The true vacuum is constructed by a linear combination of |n> vacua, and $\theta$ is usually described as 
\[

|\theta> = \Sigma_{n=-\infty}^\infy e^{in\theta} }n>

\]

therefore $\theta$ is a periodic variable of period $2\pi$. The angle $\theta$ is related to the electric dipole moment (EDM of the the neutron which if existent would violate CP symmetry. We can show this as follows.

The electric dipole moment of the neutron would have to be aligned with the spin vector (preferred axis). If we apply the T operator (time reversal symmetry) the EDM does not change sign whereas the spin does. Therefore it would violate the T symmetry because it would become a different neutron. Because of the CPT theorem which implies that every reasonable quantum field theory must be CPT invariant, CP is also violated. Actually, if there is an EDM of the neutron it has to separately violate P symmetry as well, by similar arguments as for T. We talk about the EDM of the neutron because there is a rather strict experimental limit of xx which corresponds to a limit for $\theta$ of xx.

Because the notion of very fine tuning is not popular, it is believed that $\theta$ must be exactly 0 or $\pi$, but in principle it could be anywhere between 0 and $2\pi$. Different values of $\theta$ would correspond to different theories with different coupling constants and there would be no problem with the term except the CP problem. Peccei and Quinn first intoduced the idea that $\theta$ is not a parameter of the theory but rather a dynamical variable and that different values of $\theta$ describe different energy states of the vacua of the same theory 



\end{document}