\documentclass[11pt]{book}
\usepackage{amsmath}
\usepackage{graphicx}
\usepackage{color}
\begin{document}


\chapter{Chapter 3}
\section{Parameter Space}

When the axion mass was decoupled from its coupling, this left a 2D parameter space within which to search for the axion, from less than 100 keV in principle with no lower limit. However, astrophysical arguments have been crucial in constraining the possible parameter space for ALPs.

\subsection{Astrophysical Constraints}

The evolution of stars is limited by how fast they can get rid of the energy they create in fusion. This rate is set by the opacity of the stars; for many stars it takes $10^8$ years for photons to escape from the core to the outside. One can see that if feebly interacting particles can be produced in the star, these will freely stream out and thus radiate away more energy than just the photons could, accelerating the stellar evolution and shortening the star's lifetime.

By observing the statistical distribution of stars in globular clusters, we can infer the lifetimes of stars. They are consistent with our models of stellar lifetimes (without axions), so we can set a limit on the strength of axion coupling to matter by saying that the total energy loss (if there were axions) must be within the uncertainty of the observation. For different stars, different production mechanisms are dominant; in white dwarfs, nucleon bremmstrahlung predominates; in red giants, it is Compton-like processes [NEEDSREF].

The strongest limits come from horizontal branch stars, where Primakoff processes are strongest. These exclude axions from 1 eV and below for couplings greater than $g_{a\gamma\gamma} > 10^{-10} \text{1}/\text{GeV}$.

Another limit comes from the supernova SN1987A. Neutrinos provide the dominant source of cooling for supernovas; in less than a second, the supernova radiates all of its energy through neutrinos. If axions provide another cooling method, than we would see less neutrinos. Observations at Homestake observed nineteen neutrino events, consistent with expectations [NEEDSREF]. As the primary axion-matter interaction in supernova would be nucleon bremmstrahlung, one can exclude $g_{aNN} < 10^{-11}$ 1/GeV for axions of mass less than 2 eV [NEEDSREF]. One can convert this to a constraint on the coupling to two photons if you assume the axion obeys a particular model. The two most popular models are the DFSZ and KVSZ models, which we will now describe briefly.

\subsection{Axion Models}

After the original Peccei Quinn axion was eliminated, several models were theorized to allow for axions with different symmetry breaking energy scales other than the electroweak scale. Two are prominently refered to, the DFSZ model [NEEDSREF] and KVSZ model [NEEDSREF].

The KVSZ model introduces a new heavy quark, which couples to the axion. In this model the axion does not couple to electrons or quarks at tree-level. However it ends up having a coupling to nucleons comparable to that of the DFSZ model through other corrections.

The DFSZ model supposes that all leptons have PQ charge. This model is preferred as it fits well with GUT theories.

Both models couple to photons, through the coupling to gluons.

\subsection{Direct Searches}

Astrophysical arguments constrain a fair portion of the possible phase space; giving an upper bound on the axion mass of approximately 1 meV and an upper bound on the axion to photon coupling of $10^{-10}$ 1/GeV.

Direct searches are those which attempt to directly detect signatures of axion interactions with matter in the laboratory. Most of these searches utilize the axion to two photon coupling. One experiment, dubbed "light shining through wall" [NEEDSREF], produces axions by the combination of a strong magnetic field and laser (through the interaction term in the Lagrangian $E \dot B$. By producing an opaque barrier and then looking for excess photons on the other side which could be caused by axions traversing the barrier and reconverting into photons in the presence of the magnetic field.

Axions can also be produced by the Primakoff effect wih the strong plasmas in the sun. They can be detected by reconverting them into photons on Earth in the presence of a strong magnetic field. The CAST experiment [NEEDSREF] has done this and placed the best limits, slightly better than those of the horizontal branch star limit. The current exclusion is of $g < 8.8 \times 10^{-11}$ 1/GeV.

If axions compose the cold dark matter of our galaxy, they will have an energy density with oscillations at the frequency $m_a$ with velocity dispersion most simply described as given by the virialized velocity $v \sim 10^{-3}c$. The conversion of the axion to photon in the presence of the magnetic field will then deposit all of the axion energy into the photon, creating photons with frequency $\nu \sim m_a$. By having the conversion take place in a resonant cavity with a mode resonant at $m_a$, the signal will coherently build up by a factor of the cavity quality factor Q. This technique of resonant detection, assuming a strong source of axions from dark matter, can achieve high sensitivities. This is the technique used in this thesis. The main work done in this has been by the ADMX experiment [NEEDSREF] which has excluded KVSZ axions for mass of 1.9 to 3.5 microeV. Future work is ongoing by them to cover more frequencies, possibly up to 80 $\mu$eV.

Other methods using Rydberg atoms, and methods not utilizing the Primakoff effect have been proposed or performed [NEEDSREF].
\end{document}